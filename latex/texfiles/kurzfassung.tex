\chapter{Kurzfassung}

Bilder von Bildschirminhalten sind in der heutigen digitalen, vernetzten Welt weit verbreitet.
Sie werden in vielen verschiedenen Anwendungen eingesetzt, z. B. bei Videokonferenzen, Remote-Desktop-Anwendungen und Video-Streaming.
Daher ist die Qualität dieser Bilder wichtig für das Benutzererlebnis.
In dieser Arbeit wird die Anwendung von Algorithmen zur optischen Zeichenerkennung für die Bewertung der Bildqualität von Bildschirminhalten untersucht.
Wir recherchieren modernste Texterkennungsmethoden, erstellen einen annotierten Datensatz und untersuchen die Korrelation zwischen der Leistung der optischen Zeichenerkennung und der menschlichen Beurteilung von Bildern mit verschiedenen Arten von Verzerrungen.
Die Ergebnisse deuten darauf hin, dass die optische Zeichenerkennung bei der Bewertung der Bildqualität von Bildschirminhalten für bestimmte Arten von Verzerrungen ein wertvolles Hilfsmittel sein könnte, insbesondere wenn sie in Kombination mit anderen Metriken verwendet wird.

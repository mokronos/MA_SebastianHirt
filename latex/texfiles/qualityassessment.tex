\chapter{Quality Assessment}
\label{chap:qualityassessment}

In this chapter, we begin by examining related work in the field of \gls{iqa} for natural images.
Afterwards, we describe the differences to \glspl{sci} and discuss some metrics better suited to evaluate the quality of \glspl{sci}.
Finally, we review the specific procedure used in our thesis to evaluate the performance of \gls{iqa} algorithms.

\section{Conventional Quality Assessment}

\Gls{iqa} is a research field dedicated to quantifying the quality of distorted images \cite{iqa_survey_2020}.
The term "quality" always refers to the quality of an image as perceived by the human visual system.
However, there is an exception when images are used for machine learning or other tasks that do not involve human perception.
In this case, image quality may be defined by the performance of the machine learning algorithm.
This aspect might become important when determining the appropriate compression techniques for datasets used in machine learning or identifying the types of distortions that have the most significant impact on \gls{ocr} algorithms.
Consider, for instance, an application where an \gls{ocr} algorithm is utilized to recognize text in a presentation, and the recognized text is then read aloud by a text-to-speech system for a blind person.
In this scenario, the image quality is defined by the performance of the \gls{ocr} algorithm rather than human perception.

Generally, we can divide \gls{iqa} algorithms into three categories \cite{iqa_survey_2020}: \gls{fr}, \gls{rr} and \gls{nr}.
The goal of \gls{fr} algorithms is to predict the quality of a distorted image by comparing it to the same image without distortion, called the reference image.
\Gls{rr} algorithms predict the quality of the distorted image by comparing a reduced number of features of the distorted image to the reference image.
\Gls{nr} algorithms only use the distorted image to predict its quality directly.
In this thesis, we focus on \gls{fr} algorithms, as we have access to the original images.
Another distinction can be made between the type of images that are assessed.
We can differentiate between natural images and \glspl{sci}.
Natural images can be pictures of landscapes, people or objects and are captured by image sensors while \glspl{sci} are images of screen content, containing text, graphics or UI elements and are directly recorded in a digital format.

For natural images, one common metric is the \gls{psnr} \cite{PSNRvsSSIM_2010}.
It describes the ratio of the maximum possible power of a signal and the power of corrupting noise that affects it.
When it is applied to an image, the \gls{psnr} is defined as
\begin{equation}
    \text{PSNR} = 10 \cdot \log_{10} \left( \frac{R^2}{\text{MSE}} \right),
    \label{eq:psnr}
\end{equation}
with \(R\) denoting the maximum possible pixel value of the image and the MSE describing the \gls{mse} between the distorted and the reference image.
Another common metric is the \gls{ssim} \cite{SSIM_2004}, which takes salient features of the images into account.
It combines the luminance, the contrast and the structural differences between two images into one metric.
Compared to the \gls{psnr}, the \gls{ssim} is more closely correlated with human perception of the image quality \cite{frmetric_comp_2012}.
Additionally, the \gls{ssim} was extended to the MS-\gls{ssim} \cite{ms_ssim_2003}, which incorporates details at multiple scales of the image.
This provides better coverage of the human visual system compared to the single scale approach, as it takes into account different viewing conditions and resolutions of the images.
However, the \gls{ssim} and MS-\gls{ssim} are surpassed in all criteria by the \gls{fsim} \cite{fsim_2011}.
The \gls{fsim} is based on two features, namely the phase congruency and the gradient magnitude.
Phase congruency is used to identify the importance of local structures in the image, while the gradient magnitude is used to measure the local contrast and in general the local rate of change of pixel values.
These two features are then combined into a single metric to assess the quality of the image.



\section{Screen Content Specific Quality Assessment}

\Glspl{sci} are images directly recorded in a digital format, containing text, graphics and user interface elements.
A neighboring field is document image quality assessment, which assesses the quality of scanned documents and is thus mainly concerned with text \cite{3_subj_weight_2015}.
\Glspl{sci} often contain a combination of document and natural image content.
Therefore, they differ statistically from natural and document images.
They often contain large areas of uniform color, sharp contrasts and geometric structures.
Conventional \gls{iqa} algorithms designed for natural images do not perform well in terms of correlation with the human perception of the image quality \cite{ni_scid_2017}.
Thus, different metrics are required to evaluate their the quality of \glspl{sci}.
To address this issue, some research has been conducted.

In \cite{ocr_cnn_docu_2014}, researchers use a convolutional neural network to assess the quality of documents.
This enables the automatic assessment of document quality to filter out low-quality documents, on which \gls{ocr} algorithms would perform poorly or select high quality frames from a video recording of a document.
The documents are segmented into text and non-text regions.
Then, the proposed \gls{cnn} is applied to each of the text content patches to predict quality scores, which are averaged over the whole image.
Finally, the resulting scores are analyzed for correlation with \gls{ocr} performance.
The \gls{cnn} achieves state-of-the-art performance in assessing the quality of the documents.
In \cite{text_pict_weight_2017}, the authors propose an objective \gls{iqa} metric for \glspl{sci} that considers the text and pictorial content of an image separately.
For the pictorial regions, the luminance and structural features are extracted.
On the other hand, for the regions containing text, the authors use the gradient information to predict the visual quality.
Afterwards, the two scores are weighted and combined.
The proposed method is called SFUW and shows superior performance compared to other screen content \gls{iqa} metrics.
Yang et al. \cite{3_subj_weight_2015} investigate a subjective quality score, that considers text regions, pictorial regions and the entire image separately.
The authors find that the textual regions contribute more to the overall subjective quality of an image compared to the pictorial regions.
Additionally, an objective \gls{iqa} method is proposed that uses the weighting of the subjective scores to combine objective metrics that consider the different regions into one.
Further, in \cite{ni_esim_2017}, a new screen content \gls{iqa} metric called \gls{esim} is proposed.
Compared to the other screen content metrics, \gls{esim} does not consider different regions of the image separately.
Instead, it calculates the edge contrast, the edge width and the edge direction of the whole image.
Those three features are then compared between the distorted and the reference image.
The resulting similarities are then combined into a single score by a pooling strategy, that computes the weighted average of the three metrics with the maximum edge width of the two images as its weighting factor.
Despite the missing separation of the image into different regions, \gls{esim} achieves state-of-the-art performance on the dataset used in our thesis compared to other metrics.
Similarly, in \cite{iqa_sci_gabor_2018}, the authors propose a metric that uses the Gabor filter to extract features from the image, named \gls{gfm}.
The motivation to use the Gabor filter is that it yields edge information, which is highly consistent with the human visual system.
The method compares the similarities of the two chrominance components of the distorted and the reference image.
Additionally, the Gabor filters are applied to the luminance component of both images and the similarities are compared.
Finally, the two generated similarity maps are combined by the proposed Gabor-feature-based pooling strategy.
This is done similarly to the pooling strategy of \gls{esim}.
Due to the similarity of the Gabor filter to parts of the receptive field of the human visual system, the authors used the maximum values of the two Gabor feature maps to weight the final quality metric.
The \gls{gfm} not only achieves superior performance compared to other screen content and natural image \gls{iqa} metrics, but is also less computationally complex, especially compared to the \gls{esim}.
Compared to these metrics, the \gls{ocr} methods only consider the text regions to assess the quality of the images.



\section{Evaluation Procedure of Quality Assessment Algorithms}
\label{sec:evalprocedure}

To evaluate the suitability of an objective \gls{iqa} metric, in our case the \gls{cer}, more specifically the $\text{CER}_{\text{c}}$, see \autoref{subsec:cer}, and its correlation to the human subjective score, in this thesis the \gls{mos}, see \autoref{chap:dataset}, there are three aspects to consider \cite{nonlin_fit_original_2003}\cite{iqa_survey_2020}, namely prediction consistency, prediction monotonicity, and prediction accuracy, which we describe in this section.
Since the \gls{vqeg} recommends removing nonlinearities from the data before calculating these metrics \cite{nonlin_fit_original_2003}, we describe this procedure in the next subsection.
In the following, in place of a generic quality metric, we use the $\text{CER}_{\text{c}}$ as used in the remaining parts of this thesis.

\subsection{Nonlinear Transformation}
\label{subsec:nonlinear}

To remove nonlinearities, we fit a model to the $\text{CER}_{\text{c}}$ and the \gls{mos} values.
This model is described in \cite{nonlin_fit_model_init_2000}\cite{nonlin_fit_appl_2017}.
Given the $i$-th image in our dataset, its \gls{mos} value is denoted as $\text{MOS}_{i}$, its $\text{CER}_{\text{c}}$ value as $\text{CER}_{\text{c},i}$ and its predicted \gls{mos} value as $\text{MOS}_{\text{p},i}$.
The corresponding vectors are defined as
\begin{equation}
    \mathbf{CER}_{\text{c}} =
    \begin{pmatrix}
        \text{CER}_{\text{c},1} \\
        \text{CER}_{\text{c},2} \\
        \vdots \\
        \text{CER}_{\text{c},N}
    \end{pmatrix},\text{ }
    \mathbf{MOS} =
    \begin{pmatrix}
        \text{MOS}_{1} \\
        \text{MOS}_{2} \\
        \vdots \\
        \text{MOS}_{N}
    \end{pmatrix} \text{ and }
    \mathbf{MOS}_{\text{p}} =
    \begin{pmatrix}
        \text{MOS}_{\text{p},1} \\
        \text{MOS}_{\text{p},2} \\
        \vdots \\
        \text{MOS}_{\text{p},N}
    \end{pmatrix},
\end{equation}
with $N$ being the number of images in the dataset.
This number varies, depending on the experiment, as we are selecting a subset of the dataset for each experiment.
We describe this selection in \autoref{chap:dataset} and declare which subset we use in each experiment.
Then, the model can be defined as
\begin{equation}
    \mathbf{MOS}_{\text{p}} = \frac{\beta_{1}-\beta_{2}}{1 + \text{e}^{-\left(\frac{\mathbf{CER}_{\text{c}}-\beta_{3}}{|\beta_{4}|}\right)}} + \beta_{2},
    \label{eq:nonlinear}
\end{equation}
with $\beta_{1}$, $\beta_{2}$, $\beta_{3}$, and $\beta_{4}$ denoting the parameters of the model.
Although the model in \cite{nonlin_fit_original_2003} is more recent, we could not find initial parameters for it and decided to work with the older model.
Additionally, there are other, more recent, publications \cite{ni_esim_2017, nonlin_fit_appl_2017, nonlin_fit_appl_2018, nonlin_fit_appl_2014, fsim_2011, nonlin_fit_appl_2015, doc_quality_survey_2023, iqa_database_2023, nonlin_fit_appl_2016} that use the model proposed in \cite{nonlin_fit_new_model_2006}, which does not specify initial conditions either.
In \cite{nonlin_fit_init_proof_2017}, a solution to estimating the initial parameters is proposed, which is out of scope for this thesis.
The parameters are initialized as
\begin{equation}
    \begin{aligned}
        \beta_{1} &= \max{\mathbf{CER}_{\text{c}}} \\
        \beta_{2} &= \min{\mathbf{CER}_{\text{c}}} \\
        \beta_{3} &= \overline{\text{MOS}} \\
        \beta_{4} &= 1.
    \end{aligned}
    \label{eq:nonlinear_init}
\end{equation}
The parameters are adjusted with the least squared method \cite{least_squares_1978} until the model fits the data of all the images.
The model and the $\text{CER}_{\text{c}}$ values are then used to calculate the predicted $\text{MOS}_{\text{p}}$ values.

\begin{figure}[h]
    \centering
    \includegraphics[width=\textwidth]{../exp/fit_example.pdf}
    \caption{Example of the nonlinear fitting. Data before fitting ($\text{CER}_{c}$ vs. \gls{mos}) and after fitting ($\text{MOS}_{\text{p}}$ vs \gls{mos}). Model with initial parameters and fitted parameters.}
    \label{fig:nonlinear_fit}
\end{figure}

In Figure \ref{fig:nonlinear_fit}, an example of the nonlinear fitting is depicted.
The initial data consists of some randomly generated \gls{mos} and $\text{CER}_{\text{c}}$ values.
We can see, that the model with the initial parameters does not fit the data.
The parameters are then adjusted by the least squares method \cite{least_squares_1978} to fit the model to the data.
Finally, we can observe that the fitted curve clearly fits the data better than the curve with the initial parameters.
Now, the $\text{MOS}_{\text{p}}$ values can be calculated by using the fitted model and the $\text{CER}_{\text{c}}$ values.
With the $\text{MOS}_{\text{p}}$ the following three metrics \cite{iqa_survey_2021} can be calculated.

\subsection{Pearson Correlation}
\label{subsec:pearson}

The \gls{plcc} \cite{pears_spear_2016} describes the linear correlation between two variables, normalized to the range $[-1, 1]$.
This metric is used to measure the prediction linearity and consistency of the method.
It is defined as

\begin{equation}
    \text{PLCC} = \frac{\sum_{i=1}^{N}{(\text{MOS}_{i}-\overline{\text{MOS}})(\text{MOS}_{\text{p},i}-\overline{\text{MOS}}_{\text{p}})}}{\sqrt{\sum_{i=1}^{N}{(\text{MOS}_{i}-\overline{\text{MOS}})^2}\sum_{i=1}^{N}{(\text{MOS}_{\text{p},i}-\overline{\text{MOS}}_{\text{p}})^2}}},
    \label{eq:pearson}
\end{equation}

with $\overline{\text{MOS}}$ and $\overline{\text{MOS}}_{\text{p}}$ representing the mean values of the $\mathbf{MOS}$ and $\mathbf{MOS}_{\text{p}}$ vectors respectively and $N$ representing the total number of images in the dataset.
If the \gls{plcc} is close to 1, the two vectors have a positive linear relationship, which means that if $\text{MOS}_{i}$ increases, $\text{MOS}_{\text{p},i}$ increases as well.
If the \gls{plcc} is close to -1, the two vectors have a negative linear relationship, which means that if $\text{MOS}_{i}$ increases, $\text{MOS}_{\text{p},i}$ decreases.
If the \gls{plcc} is close to 0, the two vectors have no correlation.

\subsection{Spearman Ranked Correlation}
\label{subsec:spearman}

The \gls{srcc} \cite{pears_spear_2016} describes the monotonic correlation between two variables, normalized to the range $[-1, 1]$.
Thus it is used to measure the prediction monotonicity of the method.
Compared to the \gls{plcc}, it takes the rank, or order, of the values into account, not the exact values.
The scores $\text{CER}_{\text{c},i}$ and $\text{MOS}_{i}$ are transformed into their ranks $\text{CER}_{\text{c,r},i}$ and $\text{MOS}_{\text{r},i}$ respectively, with values in the range $[1, N]$.
If for example, the first two values are tied, their rank is set to the mean, in this case $(1+2)/2 = 1.5$.
Note, that the fitting procedure does not matter for the \gls{srcc}, because the ranks between the $\text{CER}_{\text{c}}$ and $\text{MOS}$ values stay the same compared to the $\text{MOS}_{\text{p}}$ and $\text{MOS}$ values, because the fitted models are monotonic.
With these values, the \gls{srcc} is defined as

\begin{equation}
    \text{SRCC} = \frac{\sum_{i=1}^{N}{(\text{MOS}_{\text{r},i}-\overline{\text{MOS}}_{\text{r}})(\text{CER}_{\text{c,r},i}-\overline{\text{CER}}_{\text{c,r}})}}{\sqrt{\sum_{i=1}^{N}{(\text{MOS}_{\text{r},i}-\overline{\text{MOS}}_{\text{r}})^2}\sum_{i=1}^{N}{(\text{CER}_{\text{c,r},i}-\overline{\text{CER}}_{\text{c,r}})^2}}},
    \label{eq:spearman}
\end{equation}

with $\overline{\text{MOS}}_{\text{r}}$ and $\overline{\text{CER}}_{\text{c,r}}$ representing the mean values of the $\mathbf{MOS}_{\text{r}}$ and $\mathbf{CER}_{\text{c,r}}$ vectors respectively.
If the \gls{srcc} is close to 1, the two vectors have a positive monotonic relationship, which means that the rank of the $\text{CER}_{\text{c},i}$ increases, while the rank of the $\text{MOS}_{i}$ increases.
If the \gls{srcc} is close to -1, the two vectors have a negative monotonic relationship, which means that the rank of the $\text{CER}_{\text{c},i}$ increases, while the rank of the $\text{MOS}_{i}$ decreases.
If the \gls{srcc} is close to 0, the ranks of the two vectors have no correlation.
These characteristics help us to determine if the $\text{CER}_{\text{c}}$ is a good predictor for the \gls{mos}, by investigating how similar the ranks of the two metrics are.


\subsection{Root Mean Squared Error}
\label{subsec:rmse}

The \gls{rmse} is a metric that measures the average magnitude of the error between the predicted values and the actual values.
It is used to measure the prediction accuracy of the method.
In our case it is defined as

\begin{equation}
    \text{RMSE} = \sqrt{\frac{1}{N}\sum_{i=1}^{N}{(\text{MOS}_{\text{p},i} - \text{MOS}_{i})^2}}.
    \label{eq:rmse}
\end{equation}

To summarize, \gls{plcc} measures the prediction linearity and consistency, \gls{srcc} measures the prediction monotonicity and the \gls{rmse} measures the prediction accuracy.
With these, we can now determine if the $\text{CER}_{\text{c}}$ is a good predictor for the \gls{mos}.
It is a better predictor the larger the \gls{plcc} and \gls{srcc} values are and the smaller the \gls{rmse} is.


\section{Bjøntegaard Delta Rate}
\label{subsec:bdrate}

The \gls{bdrate} \cite{bdrate_original_2001}\cite{bdrate_beyond_2022} is defined as the average difference between two rate-distortion curves of two codecs.
The first curve is the reference curve and the second the test curve.
Those curves are defined by a set of points $(R_{\text{k,i}}, M_{\text{k,i}})$, where $R_{\text{k,i}}$ is the bitrate and $M_{\text{k,i}}$ is the metric, in our case the $\text{CER}_{\text{c}}$, of the image compressed by a codec $k$, with $k \in \{A,B\}$, with \gls{qp} $i$, with $i \in \{35,40,45,50\}$.
The $R_{k,i}$ values are first converted to the logarithmic scale to not bias the results towards the higher bitrates with
\begin{equation}
    r_{k,i} = \log_{10}\left(R_{k,i}\right).
    \label{eq:log_scale}
\end{equation}
Those values in combination with the corresponding $M_{k,i}$ values are then used as anchor points for an interpolation with a third order polynomial.
For our work, we use the Akima interpolation suggested in \cite{bdrate_beyond_2022}, which seems to show more accurate interpolation curves for a variety of metrics.
The resulting functions are denoted by $\hat{r}_{k}$, respectively.
The interpolation results in two curves, one for each codec, that pass through all anchor points.
\begin{figure}
    \centering
    \includegraphics[width=\textwidth]{../../exp/bjontegaard_example.pdf}
    \caption{Example of the \gls{bdrate} calculation with dummy values. Adapted from \cite{bdrate_beyond_2022}}
    \label{fig:bdrate_example}
\end{figure}
Finally, the \gls{bdrate} can be denoted as $\Delta R$ and is calculated by the integral of the difference between the two curves as
\begin{equation}
    \Delta R = 10^{\frac{1}{M_{\text{low}}-M_{\text{high}}} \int_{M_{\text{low}}}^{M_{\text{high}}} \hat{r}_{\text{B}}(M) - \hat{r}_{\text{A}}(M) \text{d}M} - 1.
    \label{eq:bdrate}
\end{equation}
The lower and upper bound of the integral are given by
\begin{equation}
    \begin{aligned}
        M_{\text{low}} = \max\left(M_{\text{A},50}, M_{\text{B},50}\right) \\
        M_{\text{high}} = \min\left(M_{\text{A},35}, M_{\text{B},35}\right).
    \end{aligned}
    \label{eq:bounds}
\end{equation}
The bounds are the maximum of the lowest quality points and the minimum of the highest quality points and can be seen in \autoref{fig:bdrate_example}.


$\Delta R$ describes the average difference between the two curves in percent.
This enables us to compare the rate-distortion curves of two codecs.
The $\text{CER}_{\text{c}}$ is a metric that describes the difference between two texts and is explored in more detail in \autoref{subsec:cer}.
In our case, we can calculate the $\text{CER}_{\text{c}}$ for each codec in two ways:
once with respect to the hand annotated text label, which we explain in \autoref{sec:dataset_labeling}, and once in relation to the prediction of the \gls{ocr} algorithms on the reference image without distortion.
Throughout this thesis, we refer to the hand annotated text label as the true \gls{gt} and the prediction of the \gls{ocr} methods on the reference image as the pseudo \gls{gt}.
Thus, we can calculate the $\Delta R$ value for two codecs with respect to the true \gls{gt} and the pseudo \gls{gt}.
These values can then be compared to evaluate, if using the \gls{ocr} algorithms as a pseudo \gls{gt} is a good estimation of the difference between codecs.
If the difference between the two values is small, then the estimation is good and we might be able to use the \gls{ocr} algorithms as a reference for future codec comparisons.
We go into more detail about the specific codecs we use for our comparisons in \autoref{sec:dataset_codec}.

In this chapter, we have outlined the evaluation procedure for the $\text{CER}_{\text{c}}$ values predicted by the \gls{ocr} algorithms.
However, to calculate the $\text{CER}_{\text{c}}$ values, we need \gls{ocr} algorithms to extract text from the images.
Thus, in the following chapter we introduce the \gls{ocr} algorithms that we use in our experiments.

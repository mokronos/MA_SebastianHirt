\chapter{Introduction}
\label{chap:Introduction}

\section{Motivation}

In today’s digital age, screen content plays a vital role in our daily lives.
From office work to entertainment, we are constantly interacting with images and videos on screens.
As such, the quality of screen content is of utmost importance.
One key aspect of screen content quality is the readability of text.
However, conventional objective image quality assessment algorithms do not directly consider text readability.
This is where text recognition algorithms come into play.

In this thesis, we will explore the application of text recognition algorithms for the assessment of screen content image quality.
By researching state-of-the-art text recognition and detection methods, generating a labeled dataset, and investigating the correlation between text recognition rates and human judgement, we aim to provide a valuable addition to conventional quality metrics.

Through a structured implementation and detailed documentation of the framework and experiments, this thesis will provide valuable insights into the potential of text recognition algorithms for screen content quality assessment.

In this thesis, I use the pronouns \textit{we} and \textit{our} to refer to myself and the larger scientific community.

\section{Research Objectives}

--- Figure out if text recognition algorithms can be used to assess the quality of screen content images.

\section{Thesis Outline}

--- do this last

First we will give an overview of the current state of the art in text recognition and detection in \autoref{chap:related_work}.
Then we will describe the methodology used in this thesis in \autoref{chap:methods}.
Afterwards, we will present the results of our experiments in \autoref{chap:evaluation}.
Finally, we will summarize and conclude our findings in \autoref{chap:conclusion}.

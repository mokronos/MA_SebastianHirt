\chapter{Methods}
\label{chap:methods}

\section{Optical Character Recognition}
\label{sec:ocr}

\section{Metrics}
\label{sec:metrics}

The \gls{ter} is defined as follows:

\begin{equation}
    \text{TER} = \frac{S + D + I}{N}
    \label{eq:ter}
\end{equation}
with \(S\) being the number of substitutions, \(D\) the number of deletions, \(I\) the number of insertions, and \(N\) the total number of characters of the text prediction.
The \gls{ter} ranges from 0 to 1, where 0 means perfect recognition and 1 means no correct characters.
Because the \gls{mos} is defined in the range 0 to 100 and 100 represents a high subjective quality, the two metrics are unintuitive to compare.
Therefore, transform the TER by subtracting it from 1 and multiplying it by 100 to get a \gls{mos}-like value, see \autoref{eq:ter2mos}.
\begin{equation}
    \text{TER} = (1 - \text{TER}_{raw}) \cdot 100.
    \label{eq:ter2mos}
\end{equation}

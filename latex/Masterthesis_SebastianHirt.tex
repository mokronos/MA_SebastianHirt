%%%% File:           lmsthesis.tex
%%%% Created:        2015-08-20
%%%% Modified:       2015-10-05
%%%% Author:         Dipl.-Ing. Marcus Laumer
%%%% Description:    This LaTeX template is supposed to be used for writing a thesis at the chair of
%%%%                 Multimedia Communications and Signal Processing (LMS) at the Friedrich-Alexander-Universität Erlangen-Nürnberg (FAU)

\documentclass[
paper=A4,               % paper format
pagesize=auto,          % provide page size for compiler
fontsize=12pt,          % font size
DIV=16,                 % type area calculation
twoside=true,           % two-sided layout for printing
BCOR=20mm,              % binding correction for printing (ask at copy shop!)
parskip=false,          % space between paragraphs
chapterprefix=true,     % prefix before chapter names: Chapter #
appendixprefix=true,    % prefix before appendix names: Appendix #
listof=totoc,           % include lists of figures and tables in TOC
bibliography=totoc,     % include bibliography in TOC
headinclude=true,       % header included in type area
footinclude=false,      % footer included in type area
headsepline=true,       % separation line between header and text
footsepline=false,      % separation line between footer and text
headings=small,         % size of headings
numbers=noenddot        % no dot after chapter heading prefixes
] {scrbook}

\usepackage{lmodern}
\usepackage[T1]{fontenc}
\usepackage[utf8]{inputenc}
%\usepackage[ngerman]{babel} % can be used for writing the thesis in German
\usepackage[onehalfspacing]{setspace}
\usepackage{amsmath,amssymb}
\usepackage{graphicx}
\usepackage{wrapfig}
% \usepackage[caption=false]{subfig}
\usepackage{subcaption}
\usepackage{booktabs}
\usepackage[printonlyused]{acronym}
\usepackage{pdfpages}

\usepackage{blindtext} % delete in final version!

% custom packages
\usepackage[backref=page]{hyperref}
% for backreference with arrow
\renewcommand\backrefxxx[3]{%
  \hyperlink{page.#1}{$\uparrow$#1}%
}
% better than acronym package, makes them auto alphabetically sorted
\usepackage[toc,nonumberlist,nomain,nopostdot,acronym,style=long]{glossaries}

\graphicspath{{../images/}{../images/analyze/}{./images/}}

\recalctypearea

% define symbols and acronyms, need to be in preamble
\newglossary*{symbols}{Symbols and Notations}
% \newglossary*{acronym}{Abbreviations and Acronyms}
\makeglossaries
% \chapter{Symbols and Notations}

% In order of appearance.\\

% \begin{tabular}{ll}
% +					&	Addition\\
% \end{tabular}
% \makenoidxglossaries

\newglossaryentry{add}
{
    type={symbols},
    name={+},
    description={Addition}
}

\newglossaryentry{sub}
{
    type={symbols},
    name={-},
    description={Subtraction}
}

% old version of acronyms.tex
% \chapter{Abbreviations and Acronyms}

% In alphabetical order.

% \begin{acronym}
%     \acro{e.g.}{exempli gratia}
%     \acro{TER}{text error rate}
%     \acro{MOS}{mean opinion score}
% \end{acronym}

\newacronym{ocr}{OCR}{optical character recognition}
\newacronym{cer}{CER}{character error rate}
\newacronym{mos}{MOS}{mean opinion score}
\newacronym{psnr}{PSNR}{peak signal-to-noise ratio}
\newacronym{mse}{MSE}{mean squared error}
\newacronym{bdrate}{BDRate}{Bjøntegaard Delta Rate}
\newacronym{hevc}{HEVC}{High Efficiency Video Coding}
\newacronym{vvc}{VVC}{Versatile Video Coding}

\glsfindwidesttoplevelname
\setglossarystyle{alttree}

\begin{document}

    % \chapter{Examples}
\label{chap:Examples}

\section{Figures}
\label{sec:Figures}

\blindtext
This is illustrated in Figure~\ref{fig:samplefigure}.

\begin{figure}[t]
    \centering
    %\includegraphics[width=0.99\textwidth]{figure}
    \rule{0.99\textwidth}{5cm}
    \caption{Sample figure with caption below.}
    \label{fig:samplefigure}
\end{figure}

\Blindtext
\blindtext
Please refer to Figures~\ref{subfig:samplesubfigure1}, \ref{subfig:samplesubfigure2}, \ref{subfig:samplesubfigure3}, and \ref{subfig:samplesubfigure4}.

\begin{figure}[t]
  \centering
  \subfloat[Caption a]{\rule{0.22\textwidth}{2cm}\label{subfig:samplesubfigure1}}\quad
  \subfloat[Caption b]{\rule{0.22\textwidth}{2cm}\label{subfig:samplesubfigure2}}\quad
  \subfloat[Caption c]{\rule{0.22\textwidth}{2cm}\label{subfig:samplesubfigure3}}\quad
  \subfloat[Caption d]{\rule{0.22\textwidth}{2cm}\label{subfig:samplesubfigure4}}
  \caption{Caption for all subfigures.}
  \label{fig:samplesubfigures}
\end{figure}


\section{Tables}
\label{sec:Tables}

\blindtext
Results can be found in Table~\ref{tab:sampletable}.

\begin{table}[t]
    \caption{Sample table with caption above.}
    \centering
    \begin{tabular}{ccc}
        \toprule
        Column 1    &   Column 2    &   Column 3\\
        \midrule
        One         &   Two         &   Three\\
        Un          &   Deux        &   Trois\\
        Eins        &   Zwei        &   Drei\\
        \bottomrule
    \end{tabular}
    \label{tab:sampletable}
\end{table}

\Blindtext

\section{Equations}
\label{sec:Equations}

\blindtext
Hence, the energy \(E\) is defined as
\begin{equation}
    E = mc^2 \text{ .}
    \label{eq:sampleequation}
\end{equation}
Thereby, the kinetic energy \(E_k\) of a moving object is defined as follows.
\begin{equation}
    \begin{aligned}
        E_k &= m_0\left(\gamma - 1\right) c^2 =\\
        &= \frac{m_0 c^2}{\sqrt{1 - \frac{v^2}{c^2}}} - m_0 c^2
    \end{aligned}
\end{equation}
Furthermore, the total momentum \(P\) of a particle and the relativistic mass \(m\) are:
\begin{align}
    P &= \frac{m_0 v}{\sqrt{1 - \frac{v^2}{c^2}}}\\
    m &= \frac{m_0}{\sqrt{1 - \frac{v^2}{c^2}}}
\end{align}

\Blindtext

    \pagenumbering{Alph} % will not be displayed
    \begin{titlepage}
    \vspace*{6ex}
    \begin{center}
        \LARGE
        Friedrich-Alexander-Universität Erlangen-Nürnberg\\[1.5ex]
        \Large
        \textbf{Lehrstuhl für Multimediakommunikation und Signalverarbeitung}\\[1.5ex]
        Prof. Dr.-Ing. André Kaup\\
        \vfill
        \LARGE
        \{Master | Bachelor\} Thesis\\[3ex]
        \textbf{<thesis title>}\\[3ex]
        <author name>\\
        \vfill
        \Large
        <month> <year>\\[1.5ex]
        \begin{tabular}{ll}
            Supervisors: & <name of prof>\\
            & <name of supervisor>
        \end{tabular}
    \end{center}
\end{titlepage}

    \cleardoublepage
    \includepdf[pages={1},scale={0.95}]{../organizing/MA_AufgabenstellungUnterzeichnet_searchable.pdf}
    \cleardoublepage
    
    %\chapter*{Erklärung}
\thispagestyle{empty}

\noindent
Ich versichere, dass ich die vorliegende Arbeit ohne fremde Hilfe und ohne Benutzung anderer als der angegebenen Quellen angefertigt habe, und dass die Arbeit in gleicher oder ähnlicher Form noch keiner anderen Prüfungsbehörde vorgelegen hat und von dieser als Teil einer Prüfungsleistung angenommen wurde.
Alle Ausführungen, die wörtlich oder sinngemäß übernommen wurden, sind als solche gekennzeichnet.

\vspace{3cm}

\begin{minipage}[t]{0.45\textwidth}
    \rule{\textwidth}{0.5pt}\\
	Ort, Datum
\end{minipage}
\hfill
\begin{minipage}[t]{0.45\textwidth}
	\rule{\textwidth}{0.5pt}\\
	<Vollständiger Name>\\
    <Adresse>
\end{minipage}

    % or
    \chapter*{Declaration}
\thispagestyle{empty}

\noindent
I confirm that I have written this thesis unaided and without using sources other than those listed and that this thesis has never been submitted to another examination authority and accepted as part of an examination achievement, neither in this form nor in a similar form.
All content that was taken from a third party either verbatim or in substance has been acknowledged as such.

\vspace{3cm}

\begin{minipage}[t]{0.45\textwidth}
    \rule{\textwidth}{0.5pt}\\
	Erlangen, \today
\end{minipage}
\hfill
\begin{minipage}[t]{0.45\textwidth}
	\rule{\textwidth}{0.5pt}\\
	Sebastian Hirt\\
    Ellingen, Rennfeld 5b
\end{minipage}


    \frontmatter
    \pagenumbering{Roman}
    \tableofcontents
    \chapter{Kurzfassung}

Abstract in German...

    \chapter{Abstract}

Screen content images are common in today's digital interconnected world.
They are used in many different applications, such as video conferencing, remote desktop applications, and video streaming.
Thus, the quality of these images is important for the user experience.
In this thesis, the application of \gls{ocr} algorithms for the assessment of screen content image quality is explored.
We research state-of-the-art \gls{ocr} methods, generate a labeled dataset and investigate the correlation between the performance of \gls{ocr} and human judgment on images with different types of distortion.
The results indicate that \gls{ocr} could be a valuable tool in screen content image quality assessment for certain types of distortions, particularly when used in combination with other metrics.


    % glossary, makes them auto alphabetically sorted
    \printglossary[type=\acronymtype]
    \printglossary[type=symbols]

    \mainmatter
    \chapter{Introduction}
\label{chap:Introduction}


In today’s digital age, screen content plays a vital role in our daily lives.
From office work to entertainment, we are constantly interacting with images and videos on screens.
Many of these images contain text, graphics and user interface elements that are not found in natural images.
As such, the quality of screen content is of utmost importance for the viewer.
One key aspect of screen content quality is the readability of text.
However, conventional objective \gls{iqa} algorithms do not directly consider text readability.
This is where \gls{ocr} algorithms come into play.

In this thesis, we will explore the application of \gls{ocr} algorithms for the assessment of screen content image quality.
We research state-of-the-art \gls{ocr} methods, generate a labeled dataset and investigate the correlation between the performance of \gls{ocr} and human judgment on images with different types of distortion.
Additionally, we investigate the feasibility of using the quality of \gls{ocr} algorithms to compare codecs.
Through a structured implementation and detailed documentation of the framework and experiments, we provide valuable insights into the potential of \gls{ocr} algorithms for screen content \gls{iqa}.
In this thesis, I use the pronouns \textit{we} and \textit{our} to refer to myself and the larger scientific community.

This thesis is structured as follows.
First, we describe which conventional \gls{iqa} methods exist for natural images and which methods are used to assess screen content images in \autoref{chap:qualityassessment}.
Further, we detail the evaluation procedure we apply to compare the performance of \gls{iqa} methods.
In \autoref{chap:ocr}, we initially describe conventional \gls{ocr} methods and then focus on the state-of-the-art \gls{ocr} methods.
Additionally, we introduce the two \gls{ocr} methods, Tesseract \gls{ocr} \cite{tesseract_legacy_2007} and EasyOCR \cite{easyocr_gitub_2020}, which we use in our experiments.
In \autoref{chap:dataset}, we detail the different types of distortion in the SCID dataset \cite{ni_esim_2017} used in this thesis and describe the labeling procedure we employ to generate text labels for the dataset.
Further, the extension of the dataset by incorporating images distorted with \gls{hevc} \cite{hevc_2012} and \gls{vvc} \cite{vvc_2021} is explained.
In \autoref{chap:evaluation} we compare the performance of Tesseract \gls{ocr} and EasyOCR for the different types of distortion, and investigate the correlation between the performance of \gls{ocr} and human judgment.
Afterwards, we investigate the feasibility of using the predictions of the \gls{ocr} algorithms to compare codecs.
Finally, we summarize and conclude our findings in \autoref{chap:conclusion}.

    \chapter{Optical Character Recognition}
\label{chap:ocr}

In this chapter, we summarize existing \gls{ocr} methods and describe the two methods we use in this thesis, EasyOCR and Tesseract \gls{ocr}, in detail.
We also introduce the \gls{cer} metric, which is used to evaluate the performance of the \gls{ocr} methods.

\section{Conventional Optical Character Recognition}

\Gls{ocr} \cite{ocr_survey_2017} generally involves the following steps to extract text from an image.
First, the image is preprocessed, which might include binarization, noise removal, and skew correction.
This step tries to improve the quality of the image from the perspective of the \gls{ocr} method.
Second, the image is segmented into individual characters, words or lines.
Next, the segmented elements are categorized by Bayesian, nearest neighbor or neural network based classifiers.
Finally, the recognized elements are postprocessed, which might include the use of multiple classifiers simultaneously and comparison of the results, incorporation of the context of the image or dictionary data to correct errors.
Over the years, many different methods have been proposed to classify the characters based on different features \cite{ocr_appl_survey_2014}\cite{ocr_appl_survey_2016}.
The main goal was to find features, that are distinctive enough for all the relevant characters to achieve a high classification performance.
These features might include bars, loops or hooks.
Applications of \gls{ocr} systems include handwriting recognition, receipt imaging, check processing in the banking industry and improvement of CAPTCHA systems, by regenerating the CAPTCHA until it cannot be recognized by the \gls{ocr} method.


\section{Neural Network Based Optical Character Recognition}

In recent years, classifiers based on neural network based methods \cite{ocr_survey_lstms_2013} have become more popular.
Especially \gls{lstm} networks are prominent, because of their leverage of the context from previous characters or regions of the image.
In general, neural network based methods perform better than conventional methods.
One of the most popular examples is Calamari \cite{ocr_calamari_2018}, which is focused on recognizing text in historical documents.
It uses the \gls{ctc} method to train a model composed of \gls{lstm} and \gls{cnn} layers.
Additionally, it includes the capability to employ multiple models simultaneously and use them to vote on the correct prediction.
To make the voting meaningful, the models can be trained with different training datasets, have different model architectures or use different base models for finetuning.
This enables a more robust final prediction and state-of-the-art performance on historical document datasets.
Another example is the Inception V3 network \cite{ocr_improved_deep_2018}, which implements a \gls{cnn} to recognize printed text in images with poor quality.
The authors leveraged fine tuning on a pre-trained base model to reduce the training time and improve the performance.
The deep learning method shows significant improvements over traditional \gls{ocr} methods, especially for low quality images.
The \gls{ocr} methods used in this thesis, EasyOCR and Tesseract \gls{ocr}, are detailed in the following sections.

\section{EasyOCR}
\label{subsec:easyocr}

\begin{figure}[h!]
    \centering
    \includegraphics[width=0.5\textwidth]{../images/external/crnn_features.png}
    \caption{EasyOCR feature sequencing for an image of the word \texttt{state}, from \cite{crnn_2015}.}
    \label{fig:easyocr_features}
\end{figure}

EasyOCR is an open source Python library for \gls{ocr} \cite{easyocr_gitub_2020}.
It leverages the CRAFT algorithm \cite{craft_2019} for text detection, utilizing a fully convolutional network structure to predict word or character boxes within an image.
For text recognition, EasyOCR adopts the None-VGG-BiLSTM-CTC architecture \cite{crnn_2015}.
The model consists of \gls{cnn} layers to extract relevant features from the input image.
These features are then transformed into sequences, as illustrated in \autoref{fig:easyocr_features}.
Then, a deep bidirectional \gls{lstm} network is used to predict a character per frame of the feature sequence.
This process generates output that may resemble something like \texttt{--ss-t-a-tt--e-}.
The benefit of using the \gls{lstm} lies in its ability to incorporate contextual information from earlier frames of the sequence to enhance the recognition accuracy.
Finally, the transcription layer uses \gls{ctc} to combine the per-frame predictions into a single text prediction.
For the example above, it does so by first removing the duplicate characters directly following each other, resulting in \texttt{--s-t-a-t--e-}.
Then, it removes the blank characters, resulting in \texttt{state}.
The blank characters are essential to allow duplicate characters to be recognized correctly, by having a blank character between them.
Optionally, this text prediction can be compared to a lexicon, which allows for additional validation and refinement.
This comparison ensures that the recognized text aligns with known words or patterns, enhancing the accuracy and reliability of the OCR results.
Throughout this thesis, we rely on version v1.6.2 of EasyOCR.

EasyOCR predicts a list of the text element bounding boxes, the confidence of the prediction and the text itself.
The bounding boxes are defined by the $(x, y)$ coordinates of their four corners.
The text elements are a mix of words and multiple words, depending on the distance between the words.
By default, the predictions are ordered from top to bottom and left to right.
The approach of ordering text in lines of text can be effective when dealing with images that contain a single text column.
However, when images have multiple text columns, enabling the paragraph mode can be more beneficial.
In this mode, the model predicts the first column's text first and then proceeds to predict the second column, rather than switching between lines of both columns.
Thus, by using paragraph mode, the model can better capture the structure and context of the text, leading to more accurate and coherent predictions for images with multiple text columns.


\section{Tesseract}
\label{subsec:tesseract}

The second \gls{ocr} method we utilize in this thesis is Tesseract \gls{ocr} \cite{tesseract_legacy_2007}\cite{tesseract_github_2023}\cite{tesseract_architecture_2016}.
Originally developed by Hewlett-Packard in the 1980s, it was open sourced in 2005 and subsequently developed by Google from 2006 to 2018.
Tesseract \gls{ocr} employs adaptive thresholding to binarize the input image, enhancing the contrast between text and background.
Following this step, the software performs page layout analysis to identify text regions within the image.
These regions are then processed by the \gls{lstm} line recognizer to generate text predictions.
Finally, the text prediction undergo correction measures to improve the accuracy of the result.
To the best of our knowledge, the literature on Tesseract \gls{ocr} focuses on older versions of the software, predating the introduction of \gls{lstm} models with version 4.0.0.
Consequently, detailed information about the \gls{lstm} line recognizer or any other replacement is not readily available.
For our experiments in this thesis, we rely on version 4.1.1 of Tesseract \gls{ocr}.
Tesseract \gls{ocr} offers multiple engine modes, namely the legacy engine only, the neural nets \gls{lstm} engine only or the Legacy + the \gls{lstm} engine.
We use the pure \gls{lstm} engine mode, as it is the default and the most recent engine mode.
Due to Tesseract \gls{ocr} being implemented in C++, we utilize the Python wrapper, \texttt{pytesseract} \cite{pytesseract_2022}, to conveniently incorporate it into our Python code.

To ensure a fair comparison, we do not preprocess the images for either \gls{ocr} method, as we are unsure of the specific preprocessing steps employed by each method itself and wish to avoid any potential bias.
For text prediction, we use the default settings for EasyOCR and Tesseract \gls{ocr}.
Comparative research between EasyOCR and Tesseract \gls{ocr}, specifically in the task of recognizing text in license plate images \cite{ocr_tess_vs_easyocr_2022}, indicates that EasyOCR generally outperforms Tesseract \gls{ocr}.
Tesseract \gls{ocr}, like EasyOCR, predicts a list of the text element bounding boxes, the confidence of the prediction and the text itself.
It has a mode to predict paragraphs as well, however we found that it orders the predictions differently than EasyOCR.
Thus, we use both methods to predict the raw bounding boxes with their text and sort them ourselves.
This is necessary because the order of the predictions impacts the \gls{cer} and we do not want to introduce any bias by using the order of one method over the other.
In \autoref{fig:order_easyocr_pred} and \autoref{fig:order_tess_pred} we illustrate the predictions of both methods for the same image and attach the number in which the bounding boxes are ordered.
EasyOCR already orders the predictions from left to right and top to bottom, as can be observed in \autoref{fig:order_easyocr_pred}.
\begin{figure}[h!]
    \centering
    \includegraphics[width=\textwidth]{../images/bbox_order_ezocr.pdf}
    \caption{Unsorted EasyOCR predictions with order information}
    \label{fig:order_easyocr_pred}
\end{figure}
Tesseract \gls{ocr}, in contrast, interprets the left menu of the website as a paragraph, even without the paragraph mode, and predicts those text elements first, as illustrated in \autoref{fig:order_tess_pred}.
\begin{figure}[h!]
    \centering
    \includegraphics[width=\textwidth]{../images/bbox_order_tess.pdf}
    \caption{Unsorted Tesseract \gls{ocr} predictions with order information}
    \label{fig:order_tess_pred}
\end{figure}
Note that the numbers for EasyOCR are lower, because it often predicts multiple words together.
To bring the predictions in line with those of EasyOCR, we apply the following procedure to the predictions of Tesseract \gls{ocr}.
First, we find the top most bounding box in the image and save the $y$ coordinates of the top half of that bounding box.
Second, we identify all bounding boxes that overlap with those saved coordinates to gather all the text elements that are in the same line.
Then, we sort those identified text elements from left to right, save them and remove them from the list of predictions.
This procedure is repeated until all text elements are sorted.
By following this procedure, the resulting order of predictions from Tesseract \gls{ocr} becomes similar to EasyOCR and can be observed in \autoref{fig:order_tess_pred_sorted}.
\begin{figure}[h!]
    \centering
    \includegraphics[width=\textwidth]{../images/bbox_order_tess_sorted.pdf}
    \caption{Sorted Tesseract \gls{ocr} predictions with order information}
    \label{fig:order_tess_pred_sorted}
\end{figure}
The major difference lies in the fact that Tesseract \gls{ocr} tends to predict single words, whereas EasyOCR often predicts multiple words together.
However, this distinction does not matter in our case since the predictions are combined into a single text prediction, by separating them with spaces, resulting in the same final text prediction.
The intention behind this procedure is to ensure a unified order of the predictions for both methods.
By doing so, we aim to prevent any disadvantage that could arise due to the order of their predictions, when comparing them to the true \gls{gt}, which we introduce in \autoref{sec:dataset_labeling}.

% However, this method is not perfect, as it is difficult to set the threshold for the $y$ coordinates correctly.
% One might include bounding boxes that do not belong to the same line.
% A better solution might be to include bounding box coordinates in the ground truth and 

\section{Character Error Rate}
\label{subsec:cer}

To evaluate the performance of \gls{ocr} methods, we use the \gls{cer}.
The \gls{cer} \cite{cer_2022} describes how many substitutions, deletions, and insertions are necessary to transform a text prediction into the correct text label.
It is defined as
\begin{equation}
    \text{CER} = \frac{I + S + D}{N},
    \label{eq:cer}
\end{equation}
with $I$ being the number of insertions, $S$ being the number of substitutions, $D$ being the number of deletions, and $N$ being the total number of characters in the text label.
The \gls{cer} ranges from 0 to $\infty$, where 0 means perfect recognition and the higher the worse the recognition.
As an example, if the text label is \texttt{hello} and the prediction is \texttt{halo}, then the \gls{cer} is $2/5 = 0.2$, due to one deletion (\texttt{l}) and one substitution (\texttt{a $\rightarrow$ e}).

In our analysis, we compare the \gls{cer} to the subjective \gls{mos}.
The details of the subjective \gls{mos} will be discussed in the following chapter.
Because the \gls{mos} is defined in the range 0 to 100 and 100 represents a high subjective quality, the two metrics are unintuitive to compare.
Therefore, we take the complement of the CER by subtracting it from 1.
Additionally we scale it by multiplying by 100 to get a \gls{mos}-like value, according to
\begin{equation}
    \text{CER}_{\text{c}} = (1 - \text{CER}) \cdot 100.
    \label{eq:cer2mos}
\end{equation}
This means that the $\text{CER}_{\text{c}}$ can be in the range -$\infty$ to 100, where 100 means perfect recognition and the lower the worse the recognition.
However, in practice, the $\text{CER}_{\text{c}}$ typically falls within the range of 0 to 100, which aligns with the scale of the \gls{mos}.
This can be illustrated by a few examples.
Let's consider the text label as \texttt{hello} and the prediction as empty.
In this case, the \gls{cer} would be $5/5 = 1$, as there are 5 insertions necessary to match the text label, and the $\text{CER}_{\text{c}}$ would be $0$.
Thus, if the prediction is shorter than the text label, the $\text{CER}_{\text{c}}$ cannot be negative.
The only scenario where a negative $\text{CER}_{\text{c}}$ can be obtained is if the prediction is longer than the text label and the prediction is incorrect.
For instance, if the text label is \texttt{hello} and the prediction is \texttt{goodbye}, the \gls{cer} would be $7/5 = 1.4$, as there are 5 substitutions and 2 deletions required to match the text label, resulting in a negative $\text{CER}_{\text{c}}$ of $-40$.
However, this case is highly unlikely, because the \gls{ocr} methods incorporate a confidence threshold, which means that they refrain from predicting a word if they lack confidence in its correctness.
Therefore, the word predictions are either empty, making the whole prediction shorter than the text label, or they have a high likelihood of being correct.

In order to apply the \gls{ocr} methods discussed in this chapter, a suitable dataset comprising images containing text is essential.
Thus, in the following chapter, we provide an overview of the dataset employed in this thesis.

    \chapter{Quality Assessment}
\label{chap:qualityassessment}

In this section, we go over the metrics used in this thesis.

\section{Conventional Quality Assessment}

Researchers used a convolutional neural network to assess the quality of documents \cite{ocr_cnn_docu_2014}.
The documents were segmented into text and non-text regions.
Then \gls{ocr} and the proposed \gls{cnn} were used to predict scores, which were analyzed for correlation.
The \gls{cnn} achieved state of the art performance in assessing the quality of the documents.
However, the authors noted that one assumption was, that the \gls{ocr} performance is directly correlated with the quality of the degraded document.
Due to the similarity of text regions in documents and screen content, we might be able to verify or disprove this assumption in this thesis.

\section{Screen Content Specific Quality Assessment}

In \cite{text_pict_weight_2017} the authors propose a objective metric that considers the text and pictorial content of an image separately.
Afterwards, the two scores are weighted and combined.

The paper \cite{3_subj_weight_2015} uses a subjective score, that is split and considers text, pictorial and the entire image separately.
The correlation between the three scores is then analyzed, weighted and used to combine them into a single score.

- No reference vs full reference, we have full reference
- most papers use ocr as a comparison, almost like ground truth, or predict when ocr is good


\section{Peak Signal-to-Noise Ratio}
\label{subsec:psnr}

The \gls{psnr} \cite{PSNRvsSSIM_2010} describes the ratio of the maximum possible power of a signal and the power of corrupting noise that affects it.
When it is applied to an image, the \gls{psnr} is defined as follows.

\begin{equation}
    \text{PSNR} = 10 \cdot \log_{10} \left( \frac{R^2}{\text{MSE}} \right),
    \label{eq:psnr}
\end{equation}

with \(R\) being the maximum possible pixel value of the image and MSE being the \gls{mse} between the original and the reconstructed image.


\section{Nonlinear Transformation}
\label{sec:nonlinear}

To evaluate the suitability of \gls{cer} as a replacement for the \gls{mos} there are three aspects to consider \cite{nonlin_fit_original_2003}\cite{iqa_survey_2020}.
Prediction accuracy, prediction monotonicity, and prediction consistency.
Before calculating metrics to measure these aspects, the \gls{vqeg} recommends removing nonlinearities from the \gls{mos} \cite{nonlin_fit_original_2003}.
The model used for the fitting can be found in \cite{nonlin_fit_model_init_2000}\cite{nonlin_fit_appl_2017} and is defined as follows.

\begin{equation}
    \text{MOS}_{\text{i,p}} = \frac{\beta_{1}-\beta_{2}}{1 + e^{-\left(\frac{\text{CER}_{\text{c,i}}-\beta_{3}}{|\beta_{4}|}\right)}} + \beta_{2},
    \label{eq:nonlinear}
\end{equation}

% \begin{equation}
%     s_{i,p} = a * (\frac{1}{2} - \frac{1}{1 + \exp{b * (x - c)}}) + d * x + e
%     \label{eq:nonlinear}
% \end{equation}

with $\text{MOS}_{p}$ representing the predicted \gls{mos} value and $\beta_{1}$, $\beta_{2}$, $\beta_{3}$, and $\beta_{4}$ the parameters of the model.

Although the model in \cite{nonlin_fit_original_2003} is more recent, we could not find initial parameters for it and decided to work with the older model.
Additionally, there are other, more recent, publications \cite{ni_esim_2017, nonlin_fit_appl_2017, nonlin_fit_appl_2018, nonlin_fit_appl_2014, nonlin_fit_appl_2011, nonlin_fit_appl_2015, doc_quality_survey_2023, iqa_database_2023, nonlin_fit_appl_2016} that use the model proposed in \cite{nonlin_fit_new_model_2006}.
In \cite{nonlin_fit_init_proof_2017} a solution to estimating the initial parameters was proposed, which was out of scope for this thesis.

- random, empirical selection can lead to local optimal solutions

The parameters are initialized as follows.
 
\begin{equation}
    \begin{aligned}
        \beta_{1} &= \max{CER} \\
        \beta_{2} &= \min{CER} \\
        \beta_{3} &= \overline{MOS} \\
        \beta_{4} &= 1
    \end{aligned}
    \label{eq:nonlinear_init}
\end{equation}
The parameters are adjusted with the least squared method until the model fits the data of all the images.
The model and the \gls{cer} values are then used to calculate the predicted \gls{mos} values $\text{MOS}_{\text{p}}$.

\begin{figure}[h]
    \centering
    \includegraphics[width=\textwidth]{../exp/fit_example.pdf}
    \caption{Example of the nonlinear fit with subjective and objective values}
    \label{fig:nonlinear_fit}
\end{figure}

In Figure \ref{fig:nonlinear_fit} an example of the nonlinear fit is depicted.
The initial data consists of some randomly generated dummy \gls{mos} and $\text{CER}_{text{c}}$ values.
The parameters are then adjusted to fit the model to the data.
Finally, we can see that the fitted curve clearly fits the data better than the curve with the initial parameters.
Now the $\text{MOS}_{\text{p}}$ values can be calculated by using the fitted model and the $\text{CER}_{\text{c}}$ values.
With the $\text{MOS}_{\text{p}}$ the three following metrics \cite{iqa_survey_2021} can be calculated.

\section{Pearson Correlation}
\label{sec:pearson}

The \gls{plcc} \cite{pears_spear_2016} describes the linear correlation between two variables, normalized to the range $[-1, 1]$.
Given the $i$th image in our dataset, its \gls{mos} value is denoted as $\text{MOS}_{\text{i}}$ and its predicted \gls{mos} value as $\text{MOS}_{\text{p,i}}$ .
The \gls{plcc} is then defined as follows.

\begin{equation}
    r_{\text{p}} = \frac{\sum_{\text{i}=1}^{N}{(\text{MOS}_{\text{i}}-\overline{\text{MOS}})(\text{MOS}_{\text{i,p}}-\overline{\text{MOS}_{\text{p}}})}}{\sqrt{\sum_{\text{i}=1}^{N}{(\text{MOS}_{\text{i}}-\overline{\text{MOS}})^2}\sum_{\text{i}=1}^{N}{(\text{MOS}_{\text{p,i}}-\overline{\text{MOS}_{\text{p}}})^2}}},
    \label{eq:pearson}
\end{equation}

with $\overline{\text{MOS}}$ and $\overline{\text{MOS}_{\text{p}}}$ representing the mean values of the $\text{MOS}$ and $\text{MOS}_{\text{p}}$ vectors respectively and $N$ the total number of images in the dataset.
$N$ is the total number of images in the images used in the experiment.
If the $r_{\text{p}}$ is close to 1, the two vectors have a positive linear relationship, which means that if $\text{MOS}_{\text{i}}$ increases, $\text{MOS}_{\text{p,i}}$ increases as well.
If the $r_{\text{p}}$ is close to -1, the two vectors have a negative linear relationship, which means that if $\text{MOS}_{\text{i}}$ increases, $\text{MOS}_{\text{p,i}}$ decreases.
If the $r_{\text{p}}$ is close to 0, the two vectors have no correlation at all.
We are using the sample form of the correlations, because we want to make a statement about a much larger population of images, not just our dataset.

\section{Spearman Ranked Correlation}
\label{sec:spearman}

The \gls{srcc} \cite{pears_spear_2016} describes the monotonic correlation between two variables, normalized to the range $[-1, 1]$.
Compared to the \gls{plcc}, it mainly takes the order/rank of the values into account, not the exact values.
The scores $\text{CER}_{\text{c,i}}$ and $\text{MOS}_{\text{i}}$ are transformed into their ranks $\text{CER}_{\text{c,i,r}}$ and $\text{MOS}_{\text{i,r}}$ respectively with values in the range $[1, N]$.
If for example, the first two values are tied, their rank is set to the mean, in this case $(1+2)/2 = 1.5$.
With these values, the \gls{srcc} is defined as follows.

\begin{equation}
    r_{\text{s}} = \frac{\sum_{\text{i}=1}^{N}{(\text{MOS}_{\text{r,i}}-\overline{\text{MOS}_{\text{r}}})(\text{CER}_{\text{c,r,i}}-\overline{\text{CER}_{\text{c,r}}})}}{\sqrt{\sum_{\text{i}=1}^{N}{(\text{MOS}_{\text{r,i}}-\overline{\text{MOS}_{\text{r}}})^2}\sum_{\text{i}=1}^{N}{(\text{CER}_{\text{c,r,i}}-\overline{\text{CER}_{\text{c,r}}})^2}}},
    \label{eq:spearman}
\end{equation}

with $\overline{\text{MOS}_{\text{r}}}$ and $\overline{\text{CER}_{\text{c,r}}}$ representing the mean values of the $\text{MOS}_{\text{r}}$ and $\text{CER}_{\text{c,r}}$ vectors respectively.
If the $r_{\text{s}}$ is close to 1, the two vectors have a positive monotonic relationship, which means that the rank of $\text{CER}_{\text{c,i}}$ increases, while the rank of $\text{MOS}_{\text{i}}$ increases.
If the $r_{\text{s}}$ is close to -1, the two vectors have a negative monotonic relationship, which means that the rank of $\text{CER}_{\text{c,i}}$ increases, while the rank of $\text{MOS}_{\text{i}}$ decreases.
If the $r_{\text{s}}$ is close to 0, the ranks of the two vectors have no correlation at all.
These characteristics will help us to determine if the $\text{CER}_{\text{c}}$ is a good alternative for the \gls{mos}, by checking how similar the ranks of the two metrics are.


\section{Root Mean Squared Error}
\label{sec:rmse}

The \gls{rmse} is a metric that measures the average magnitude of the error between the predicted values and the actual values.
In our case it is defined as follows.

\begin{equation}
    \text{RMSE} = \sqrt{\frac{1}{N}\sum_{i=1}^{N}{(\text{MOS}_{\text{p,i}} - \text{MOS}_{\text{i}})^2}}
    \label{eq:rmse}
\end{equation}

From these metrics, $r_{\text{p}}$ measures the prediction linearity and consistency, $r_{\text{s}}$ measures the prediction monotonicity and the \gls{rmse} measures the prediction accuracy.
With these, we can now determine if the $\text{CER}_{\text{c}}$ is a good alternative for the \gls{mos}.
It is a better alternative the larger $r_s$ and $r_p$ values are, and the smaller the \gls{rmse} is.

    \chapter{Dataset}
\label{chap:dataset}
In this chapter, we present the dataset used in our work.
We use the \gls{scid} dataset \cite{ni_esim_2017}\footnote{The dataset can be downloaded here: https://eezkni.github.io/publications/ESIM.html.} as the base for our experiments.
The \gls{scid} dataset is suitable for our work, since the images contain text on \glspl{sci}, different distortion levels and \gls{mos} values for each image.
Among the available screen content datasets mentioned in the literature \cite{iqa_survey_2020}, we find that other options are either inaccessible or fail to fulfill all the necessary criteria we require for our research.
An overview of the 40 reference images of the \gls{scid} dataset can be seen in \autoref{fig:dataset_overview}.
Additionally, the dataset contains 1800 distorted images, which we discuss in the next section in detail.
Further, \gls{mos} values are included for each of the distorted images, which represent the perceived quality by a human observer.
The subjective tests to obtain the \gls{mos} values were conducted using the double stimulus method, involving the following steps.
First, the reference image was shown to the candidates for 10 seconds, followed by a mid-gray screen.
Afterwards, the distorted image was shown for 10 seconds.
Finally, the candidates were asked to rate the distorted image's quality compared to the reference image on a 5-point scale, 1 being the worst and 5 being the best.
These scores are then converted to a \gls{mos} value for each image between 0 and 100, with 100 representing the highest quality.
\begin{figure}
    \centering
    \includegraphics[width=\textwidth]{reference_images}
    \caption{The 40 references images of the dataset.}
    \label{fig:dataset_overview}
\end{figure}


\section{Distortion types}
\label{sec:dataset_distortion_types}


The 1800 distorted images are generated from the 40 reference images \cite{ni_esim_2017}.
They are distorted with 9 different distortion types, each with 5 different distortion quality levels.
In \autoref{tab:distortion_types}, we list the distortion types with a short description.
\begin{table}
\centering
\caption{Overview of the distortion types used in the dataset.}
\begin{tabular}{|p{6cm}|c|p{6cm}|}
\hline
\textbf{Distortion Type} & \textbf{Abbreviation} & \textbf{Description} \\
\hline
\hline
Gaussian Noise & GN & Addition of noise to an image using a Gaussian distribution \\
\hline
Gaussian Blur & GB & Blurring of an image using a Gaussian kernel \\
\hline
Motion Blur & MB & Blurring of an image due to movement of the camera or the object \\
\hline
Contrast Change & CC & Change in the contrast of an image \\
\hline
Joint Photographic Experts Group & JPEG & Image compression standard \\
\hline
Joint Photographic Experts Group 2000 & JPEG2000 & Image compression standard \\
\hline
Color Saturation Change & CSC & Changes in the color saturation of an image \\
\hline
High Efficiency Video Coding-Screen Content Coding & HEVC-SCC & Video compression standard for screen content \\
\hline
Color Quantization with Dithering & CQD & Reduction of colors available in an image \\
\hline
\end{tabular}
\label{tab:distortion_types}
\end{table}
The images distorted by \gls{gn} have noise added with zero mean and standard deviations of $0.001$, $0.005$, $0.01$, $0.05$ and $0.1$ for each quality level, respectively.
The images distorted by \gls{gb} are blurred with a Gaussian kernel.
The size of the kernel is $5\times5$ with standard deviations of $0.58$, $0.76$, $0.96$, $1.2$ and $2.1$ for each quality level, respectively.
Images distorted by \gls{mb} are blurred with a motion kernel, which simulates motion blur.
The parameter, which controls the degree of angle in a counter-clockwise direction, is set to zero and the parameter, which determines the length of the movement of the simulated camera, is set to $2$, $3.4$, $4$, $5.5$ and $6.4$, respectively.
The \gls{cc} distortion scales certain pixel values in the reference image to new values to change the contrast.
The scaling is applied for the ranges $[0,1] \rightarrow [0.3,0.5]$, $[0,1] \rightarrow [0.1,0.7]$, $[0.1,0.8] \rightarrow [0.1,0.9]$, $[0.2,0.8] \rightarrow [0.1,0.8]$ and $[0.2,0.7] \rightarrow [0,1]$, respectively.
So for the first quality level, all pixel values (from 0 to 1) are scaled to values between 0.3 and 0.5 of the maximum pixel intensity.
For the \gls{jpeg} compression, the images are compressed by the image compression algorithm with quality factors $75$, $35$, $18$, $8$ and $5$, respectively.
The \gls{jpeg}2000 compression is applied with compression ratios of $0.08$, $0.045$, $0.02$, $0.015$ and $0.01$, respectively.
The \gls{csc} distortion keeps the luminance component of the images constant, but scales the chrominance components by the factors $0.96$, $0.72$, $0.58$, $0.42$ and $0.1$, respectively.
The \gls{hevc}-SCC distortion is applied by using the \gls{hevc} codec with the \gls{scc} configuration on the images with the \glspl{qp} set to $16$, $36$, $40$, $42$ and $48$, respectively.
We are unsure which exact software version was used for the \gls{hevc}-\gls{scc} encoding.
The \gls{cqd} distortion is applied by reducing the number of colors available in the image to $30$, $28$, $25$, $10$ and $5$, respectively.
More detailed descriptions of the implementation of the distortions can be found in the supporting file included with the dataset.
The different distortion types, at their most severe quality level, can be seen in \autoref{fig:distortion_types}, applied to the image in \autoref{fig:img29}.

\begin{figure}
    \centering
    \includegraphics[width=0.7\textwidth]{../../data/raw/scid/ReferenceSCIs/SCI29.png}
    \caption{Reference image SCI29}
    \label{fig:img29}
\end{figure}

\begin{figure}
    \centering
    \begin{subfigure}[b]{0.42\textwidth}
        \includegraphics[width=\textwidth]{../../data/raw/scid/DistortedSCIs/SCI29_1_5.png}
        \caption{Gaussian Noise}
        \label{fig:distortion_type_1}
    \end{subfigure}
    \hfill
    \begin{subfigure}[b]{0.42\textwidth}
        \includegraphics[width=\textwidth]{../../data/raw/scid/DistortedSCIs/SCI29_2_5.png}
        \caption{Gaussian Blur}
        \label{fig:distortion_type_2}
    \end{subfigure}
    \newline
    \begin{subfigure}[b]{0.42\textwidth}
        \includegraphics[width=\textwidth]{../../data/raw/scid/DistortedSCIs/SCI29_3_5.png}
        \caption{Motion Blur}
        \label{fig:distortion_type_3}
    \end{subfigure}
    \hfill
    \begin{subfigure}[b]{0.42\textwidth}
        \includegraphics[width=\textwidth]{../../data/raw/scid/DistortedSCIs/SCI29_4_5.png}
        \caption{Contrast Change}
        \label{fig:distortion_type_4}
    \end{subfigure}
    \newline
    \begin{subfigure}[b]{0.42\textwidth}
        \includegraphics[width=\textwidth]{../../data/raw/scid/DistortedSCIs/SCI29_5_5.png}
        \caption{JPEG Compression}
        \label{fig:distortion_type_5}
    \end{subfigure}
    \hfill
    \begin{subfigure}[b]{0.42\textwidth}
        \includegraphics[width=\textwidth]{../../data/raw/scid/DistortedSCIs/SCI29_6_5.png}
        \caption{JPEG2000 Compression}
        \label{fig:distortion_type_6}
    \end{subfigure}
    \newline
    \begin{subfigure}[b]{0.42\textwidth}
        \includegraphics[width=\textwidth]{../../data/raw/scid/DistortedSCIs/SCI29_7_5.png}
        \caption{Color Saturation Change}
        \label{fig:distortion_type_7}
    \end{subfigure}
    \hfill
    \begin{subfigure}[b]{0.42\textwidth}
        \includegraphics[width=\textwidth]{../../data/raw/scid/DistortedSCIs/SCI29_8_5.png}
        \caption{HEVC Screen Content Coding}
        \label{fig:distortion_type_8}
    \end{subfigure}
    \newline
    \begin{subfigure}[b]{0.42\textwidth}
        \includegraphics[width=\textwidth]{../../data/raw/scid/DistortedSCIs/SCI29_9_5.png}
        \caption{Color Quantization with Dithering}
        \label{fig:distortion_type_9}
    \end{subfigure}
    \caption{SCI 29 distorted by 9 different distortion types at the most severe level.}
    \label{fig:distortion_types}
\end{figure}

The impact of different distortion types on text within an image is evident.
Among the various distortions, alterations in contrast or color have minimal effect on text legibility for human readers.
Conversely, distortions such as \gls{gn}, \gls{gb} and \gls{mb} can render the text completely unreadable to the human eye.
We might expect that the distortions that affect the text for the human visual system the most, will also affect the \gls{ocr} the most.
It is to note, that all distortions are monotonically decreasing in their severity with the quality level\footnote{It should be noted that referring to it as "quality level" might be somewhat counterintuitive, as the highest "quality level" (level 5) corresponds to the worst quality of the distorted image.} from 1 to 5.
The only exception is \gls{cc}.
As illustrated in \autoref{fig:cc_levels}, \gls{cc} does not display a clear pattern of variation from low contrast to high contrast, or any similar trend.
Understanding this behavior is important for the analysis of the trends of the \gls{mos} over the different quality levels.

\begin{figure}
    \centering
    \begin{subfigure}[b]{0.18\textwidth}
        \includegraphics[width=\textwidth]{../../data/raw/scid/DistortedSCIs/SCI01_4_1.png}
    \end{subfigure}
    \hfill
    \begin{subfigure}[b]{0.18\textwidth}
        \includegraphics[width=\textwidth]{../../data/raw/scid/DistortedSCIs/SCI01_4_2.png}
    \end{subfigure}
    \hfill
    \begin{subfigure}[b]{0.18\textwidth}
        \includegraphics[width=\textwidth]{../../data/raw/scid/DistortedSCIs/SCI01_4_3.png}
    \end{subfigure}
    \hfill
    \begin{subfigure}[b]{0.18\textwidth}
        \includegraphics[width=\textwidth]{../../data/raw/scid/DistortedSCIs/SCI01_4_4.png}
    \end{subfigure}
    \hfill
    \begin{subfigure}[b]{0.18\textwidth}
        \includegraphics[width=\textwidth]{../../data/raw/scid/DistortedSCIs/SCI01_4_5.png}
    \end{subfigure}
    \caption{Distorted image 1 with different levels of \gls{cc}, quality levels from left to right: 1, 2, 3, 4, 5.}
    \label{fig:cc_levels}
\end{figure}

\section{Labeling}
\label{sec:dataset_labeling}

Since we want to evaluate the \gls{ocr} algorithms on theses images, we need a true \gls{gt} in the form of a text label for each image, which are not contained in the dataset \cite{ni_esim_2017}.
The following procedure is used to create the true \gls{gt} for each image.
We start by locating the topmost word in the image.
Then, we identify if this word is part of a line.
If it is, we record the text of the entire line.
Afterwards, we move to the next line or word and repeat the process until all text elements are recorded.
Finally, we combine all the recorded text elements into the full true \gls{gt} by separating them with spaces.
In this process, we ignore paragraphs and only consider the vertical position of the lines.
This true \gls{gt} aligns with the prediction order of the \gls{ocr} algorithms, as discussed in \autoref{subsec:tesseract}.
To avoid introducing bias towards any specific \gls{ocr} algorithm regarding the order of text elements, we made the decision not to utilize one of the algorithms for predicting the text and then correcting it to create the true \gls{gt}, but fully label them ourself.

% --- In ground truth margins of what is in the same line are way slimmer.

% --- The important thing is however that both algorithms have similar rules, so that they can be compared fairly.

\section{Analysis}
\label{sec:dataset_analysis}


Before we continue with our main experiments, we give a short analysis of the dataset.
First, we select a subset of images for our experiments.
Some of the images contain no text at all or only numbers.
Others contain some text, but have a large focus on graphical objects besides the text.
This makes them less suitable for a comparison between the $\text{CER}_{\text{c}}$, which evaluates text, with the \gls{mos}, which evaluates the whole image.
Due to these factors, for our experiments that involve the \gls{mos}, we select the images with
\begin{equation}
    i \in \{1, 2, 3, 4, 5, 6, 7, 8, 11, 12, 15, 18, 20, 21, 24, 29\},
    \label{eq:mos_images}
\end{equation}
as their main focus lies on text elements, instead of graphical objects.
This selection process is subjective, so it might be more reasonable to use all images and filter out outliers later.
Additionally, even if an image only has one small text element, the $\text{CER}_{\text{c}}$ might still be a good estimation of the \gls{mos}, if the distortion affects the text in the same way as the rest of the image.
This however, is not the case for certain distortions, like \gls{jpeg} or other compression algorithms.

\begin{figure}
    \begin{subfigure}[b]{0.45\textwidth}
        \centering
        \includegraphics[width=\textwidth]{../../images/cer_mos_overview_gt_tess.pdf}
        \caption{CER vs MOS for Tesseract \gls{ocr} and the true \gls{gt}.}
        \label{fig:cer_vs_mos_gt_tess}
    \end{subfigure}
    \hfill
    \begin{subfigure}[b]{0.45\textwidth}
        \centering
        \includegraphics[width=\textwidth]{../../images/cer_mos_overview_gt_ezocr.pdf}
        \caption{CER vs MOS for EasyOCR and the true \gls{gt}.}
        \label{fig:cer_vs_mos_gt_ocr}
    \end{subfigure}
    \newline
    \begin{subfigure}[b]{0.45\textwidth}
        \centering
        \includegraphics[width=\textwidth]{../../images/cer_mos_overview_ref_tess.pdf}
        \caption{CER vs MOS for Tesseract \gls{ocr} and the pseudo \gls{gt}.}
        \label{fig:cer_vs_mos_ref_tess}
    \end{subfigure}
    \hfill
    \begin{subfigure}[b]{0.45\textwidth}
        \centering
        \includegraphics[width=\textwidth]{../../images/cer_mos_overview_ref_ezocr.pdf}
        \caption{CER vs MOS for EasyOCR and the pseudo \gls{gt}.}
        \label{fig:cer_vs_mos_ref_ocr}
    \end{subfigure}
    \caption{$\text{CER}_{\text{c}}$ vs \gls{mos} for Tesseract \gls{ocr} and EasyOCR, with the true \gls{gt} and the pseudo \gls{gt}.}
    \label{fig:cer_vs_mos_overview}
\end{figure}

In \autoref{fig:cer_vs_mos_overview} the $\text{CER}_{\text{c}}$ is plotted against the \gls{mos} for both \gls{ocr} algorithms and both \glspl{gt}.
Generally, we observe that the \gls{mos} ranges from 20 to 80 for all figures.
On the other hand, the $\text{CER}_{\text{c}}$ spans from 0 to 100 for all figures.
Comparing Tesseract \gls{ocr} with EasyOCR for the true \gls{gt}, we observe that the $\text{CER}_{\text{c}}$ distribution for Tesseract \gls{ocr} is more spread out with more lower $\text{CER}_{\text{c}}$ values compared to EasyOCR
Additionally, there are some points with zero $\text{CER}_{\text{c}}$ for Tesseract \gls{ocr}.
This implies that EasyOCR performs better in general and that Tesseract \gls{ocr} struggles with some distortions and fails to predict anything.
We notice similar behavior when comparing the pseudo \glspl{gt} for both \gls{ocr} algorithms, although the $\text{CER}_{\text{c}}$ values are generally higher.
This can be attributed to the fact that the predictions contain no additional errors due to positioning of the text elements and are generally closer to the pseudo \gls{gt} compared to the true \gls{gt}.
Lastly, it is worth noting that there are minimal occurrences of high \gls{mos} values paired with low $\text{CER}_{\text{c}}$ values, which is evident in the top left sections of all figures.
In the following section we detail the extension of the dataset with compressed images that are used to compare two video codecs.


\section{Extension of the Dataset with Images Distorted by Compression Methods}
\label{sec:dataset_codec}

In this section, we first introduce the two video codecs, the \gls{hevc} and the \gls{vvc}.
We further, explain how the two codecs are adjusted for \glspl{sci} by using screen content extensions.
Finally, we use these codecs for the extension of the dataset by encoding the reference images.

\subsection{High Efficiency Video Coding}
\label{subsec:hevc}

The \gls{hevc} \cite{hevc_2012} is one of the newest video codecs.
It is the successor of the H.264/MPEG-4 AVC codec.
The main improvements were the leveraging of parallel processing architecture in modern devices and addressing higher resolutions.
The codec uses the conventional approach of dividing the image into block shaped regions.
The information about the block size is added to the bit stream sent to the decoder.
The first image of a video sequence uses intraframe prediction, which uses information from neighboring blocks to predict the information in the current block.
For further frames, interframe prediction is used, which leverages the difference from the previous frame to encode the current frame.
This improves coding efficiency, since the difference between frames is usually cheaper to encode than a whole new frame.
However, in our case we only encode single images, so the codecs only use the intraframe prediction.
After predicting the current block, the residual, which represents the difference between the prediction and the original block, is transformed by a linear spatial transform to generate the transform coefficients.
Those are scaled, quantized and entropy coded to further reduce the bit rate.
The prediction information, the transform coefficients and all other required information is then sent to the decoder.
The decoder uses that information to predict the current block as well by replicating the encoder.
Afterwards, the transform coefficients can be reconstructed and used to approximate the residual of the block.
The residual is then added to the prediction to reconstruct the original block.
Further, it employs a number of new techniques over its predecessors to provide around 50\% bit rate savings for equivalent quality.
To conform to the special characteristics of \glspl{sci}, a screen content extension was developed for the \gls{hevc} \cite{hevc_scc_2015}.
We will expand on this in the following subsection after we introduce the \gls{vvc}, as they share some similarities.
For this thesis, we use version 16.21+SCM-8.8 of the HM reference software \cite{hevc_software_2020} to encode the images with the \gls{hevc} codec.
The default \cite{config_hevc_2013} and \gls{scc} \cite{config_hevc_scc_2015} configurations used for encoding are applied according to the common test conditions for color space RGB 444.

\subsection{Versatile Video Coding}
\label{subsec:vvc}

The \gls{vvc} \cite{vvc_2021} is the successor of the \gls{hevc} and one of the most recent video codecs.
Compared to the \gls{hevc}, it introduces combined inter-/intraframe prediction, luma mapping with chroma scaling and additional loop filters.
Additionally, while most implementations of the \gls{hevc} are only able to use square block sizes, the \gls{vvc} supports rectangular block sizes as well, which enables more efficient coverage of regions that can be encoded efficiently.
It aims to reach another 50\% bit rate savings compared to the \gls{hevc} for equivalent quality.
Further, the versatility of the codec enables it to be used for a wider range of applications, including $360^{\circ}$ immersive video, high dynamic range, adaptive streaming with resolution changes and many more.
For this thesis, we use version 17.2 of the VTM reference software \cite{vvc_software_2022} to encode the images with the \gls{vvc} codec.
The default and \gls{scc} \cite{config_vvc_both_2020} configurations used for encoding are applied according to the common test conditions for color space RGB 444.

For \glspl{sci} there are some additional tools \cite{vvc_2021} to improve the performance of the codecs due to the different characteristics of the images.
One such tool is the palette mode, which uses a reduced number of colors to encode blocks of images, because \glspl{sci} generally contain a limited amount of colors in local regions.
This tool exists for the \gls{hevc}, but is further improved in the \gls{vvc}.
Another tool is the intra-picture block copy, which enables the codecs to use a copy of a block as the prediction for another block,
It leverages the fact that \glspl{sci} often contain repeated patterns, for instance in the form of UI elements or large uniformly colored regions.
In the \gls{hevc} screen content extension, this tool is able to copy blocks from the same frame from further away, while in the \gls{vvc} the complexity is reduced by restricting the copying to neighboring blocks.
\begin{figure}
    \centering
    \begin{subfigure}[b]{0.45\textwidth}
        \includegraphics[width=\linewidth]{../images/codec_hm_default_diff_50_SCI4.png}
        \caption{Default configuration}
        \label{fig:codec_hm_default_diff_50_SCI4}
    \end{subfigure}
    \hfill
    \begin{subfigure}[b]{0.45\textwidth}
        \includegraphics[width=\linewidth]{../images/codec_hm_scc_diff_50_SCI4.png}
        \caption{\gls{scc} configuration}
        \label{fig:codec_hm_scc_diff_50_SCI4}
    \end{subfigure}
    \caption{Normalized absolute pixel differences between the reference image and the \gls{hevc} encoded images with the default and \gls{scc} configurations for Image 4.}
    \label{fig:codec_hm_diff_50_SCI4}
\end{figure}
The improvements from the screen content tools for \gls{hevc} are evident when observing \autoref{fig:codec_hm_diff_50_SCI4}.
The figures depict the normalized absolute pixel differences between a reference image and its corresponding encoded image, with brighter pixels representing a larger difference.
Notably, the absolute pixel differences, representing the coding error, are reduced for the \gls{scc} configuration compared to the default configuration, particularly in the text regions.
\begin{figure}
    \centering
    \begin{subfigure}[b]{0.45\textwidth}
        \includegraphics[width=\linewidth]{../images/codec_vtm_default_diff_50_SCI4.png}
        \caption{Default configuration}
        \label{fig:codec_vtm_default_diff_50_SCI4}
    \end{subfigure}
    \hfill
    \begin{subfigure}[b]{0.45\textwidth}
        \includegraphics[width=\linewidth]{../images/codec_vtm_scc_diff_50_SCI4.png}
        \caption{\gls{scc} configuration}
        \label{fig:codec_vtm_scc_diff_50_SCI4}
    \end{subfigure}
    \caption{Normalized absolute pixel differences between the reference image and the \gls{vvc} encoded images with the default and \gls{scc} configurations for Image 4.}
    \label{fig:codec_vtm_diff_50_SCI4}
\end{figure}
A similar, albeit more subtle difference is observable for \gls{vvc} in \autoref{fig:codec_vtm_diff_50_SCI4}.


It is to note, that we use these codecs on images instead of videos, which implies that we are not leveraging the full potential of the videos codecs.
To extend the dataset we encode the reference images with the default and the \gls{scc} configuration of the codecs.
The most important difference to the other distorted images is that there are no subjective scores available for these images.
However, we can use more images for the comparison of the codecs, as we do not need to select based on too much focus on graphical objects over the text elements.
For the experiments related to the codecs we select the images with
\begin{equation}
    i \in \{1, 2, 3, 4, 5, 6, 7, 8, 9, 11, 12, 13, 15, 16, 18, 19, 20, 21, 22, 23, 24, 25, 27, 29\}.
\end{equation}
The common test condition \glspl{qp} $\in \{22, 27, 32, 37\}$ result in no significant changes in the $\text{CER}_{\text{c}}$.
Therefore, following the approach of previous researchers in \cite{ultra_low_bitrate_2022}, we encode these images with \glspl{qp} $\in \{35, 40, 45, 50\}$ for both codecs.
We subsequently employ the \gls{ocr} algorithms to extract text from these images, enabling us to compute the $\text{CER}_{\text{c}}$.
Afterwards, we can visualize rate-distortion curves and calculate the \gls{bdrate}, as described in \autoref{subsec:bdrate}.

In summary, the original dataset contains a \gls{mos} for each image with various distortions.
To evaluate the performance of the \gls{ocr} algorithms, we create text labels for each of the images.
Further, we expand the dataset by encoding select reference images with the \gls{hevc} and \gls{vvc} codecs.
With all the necessary components to describe the experiments now in place, we proceed to evaluate and discuss our results in the subsequent chapter.

    \chapter{Evaluation}
\label{chap:evaluation}

\begin{enumerate}
\item Research state-of-the-art text recognition and detection methods.

\item Generate a labeled dataset to evaluate the efficiency of the researched algorithms
   on screen content data.

\item Available datasets with subjective quality scores will be utilized to investigate
   the correlation between text recognition rates and human judgement.

\item Since most datasets do not contain textual ground truth information,
   investigate the feasibility of using recognized text from pristine images as ground truth instead.

\end{enumerate}

\section{Efficiency of ocr algorithms}

Generate a labeled dataset to evaluate the efficiency of the researched algorithms
on screen content data.

\begin{itemize}
\item Easy ocr generally performs well even on distorted images
\item for motion blur the performance is worse for the 2 worst quality levels for most images
\item image 4 performs better for worse quality, because gt doesn't contain text on coin
\end{itemize}

% \begin{figure}[h]
% \centering
% \includegraphics[width=0.9\textwidth]{../../data/raw/scid/DistortedSCIs/SCI04_8_1.bmp}
% \includegraphics[width=0.9\textwidth]{../../data/raw/scid/DistortedSCIs/SCI04_8_5.bmp}
% \caption{Image 4 with quality 1 and 5.}
% \label{fig:img4}
% \end{figure}

\section{Comparison of ocr CER score and MOS}

Available datasets with subjective quality scores will be utilized to investigate
the correlation between text recognition rates and human judgement.

\subsection{Easy Ocr}

\begin{itemize}
\item no clear correlation, ocr not getting substantially worse with worse quality \autoref{fig:sub29} and \autoref{fig:sub3}
\item transformation via fitted model might help
\end{itemize}

\begin{figure}[h]
\centering
\includegraphics[width=0.9\textwidth]{../../images/analyze/mos_ter_ezocr_sub_img29.pdf}
\caption{Performance of easyocr on image 29.}
\label{fig:sub29}
\end{figure}

\begin{figure}[h]
\centering
\includegraphics[width=0.9\textwidth]{../../images/analyze/mos_ter_ezocr_img29.pdf}
\caption{Performance of easyocr on image 29.}
\label{fig:img29}
\end{figure}

\begin{figure}[h]
\centering
\includegraphics[width=0.9\textwidth]{../../images/analyze/mos_ter_fit_ezocr_img29.pdf}
\caption{Performance of easyocr on image 29 with fitted values.}
\label{fig:img29_fit}
\end{figure}

\begin{figure}[h]
\centering
\includegraphics[width=0.9\textwidth]{../../images/analyze/mos_ter_fit_ezocr_sub_img29.pdf}
\caption{Performance of easyocr on image 29 with fitted values.}
\label{fig:sub29_fit}
\end{figure}

\begin{figure}[h]
\centering
\includegraphics[width=0.9\textwidth]{../../images/analyze/mos_ter_ezocr_sub_img3.pdf}
\caption{Performance of easyocr on image 3.}
\label{fig:sub3}
\end{figure}

\section{Usage of recognized text as ground truth}

Since most datasets do not contain textual ground truth information,
in a further step, Mr Hirt will investigate the feasibility of
using recognized text from pristine images as ground truth instead.

    \chapter{Conclusion}
\label{chap:Conclusion}

Conclusion...



    \appendix
    \chapter{First Appendix}
\label{chap:FirstAppendix}

Appendix (optional).


    \backmatter
    \listoffigures
    \listoftables
    \bibliographystyle{IEEEtran}
    \bibliography{bibfiles/references}
    \chapter*{Curriculum Vitae}
\thispagestyle{empty}

\begin{wrapfigure}{r}{3.5cm}
    %\includegraphics[width=3.5cm]{photo}
    \rule{3.5cm}{4.5cm}
\end{wrapfigure}

Short CV of the author.
\blindtext

\end{document}
